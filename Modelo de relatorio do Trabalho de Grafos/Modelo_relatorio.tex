\documentclass[12pt,fleqn]{article}
%\usepackage {psfig,epsfig} % para incluir figuras em PostScript
\usepackage{amsfonts,amsthm,amsopn,amssymb,latexsym}
\usepackage{graphicx}
\usepackage[T1]{fontenc}
\usepackage[brazil]{babel}
%\usepackage{geometry}
\usepackage[latin1]{inputenc}
\usepackage[intlimits]{amsmath}
%alguns macros
% \newcommand{\R}{\ensuremath{\mathbb{R}}}
% \newcommand{\Rn}{{\ensuremath{\mathbb{R}}}^{n}}
% \newcommand{\Rm}{{\ensuremath{\mathbb{R}}}^{m}}
% \newcommand{\Rmn}{{\ensuremath{\mathbb{R}}}^{{m}\times{n}}}
% \newcommand{\contcaption}[1]{\vspace*{-0.6\baselineskip}\begin{center}#1\end{center}\vspace*{-0.6\baselineskip}}
%=======================================================================
% Dimensıes da p·gina
\usepackage{a4}                       % tamanho da p·gina
\setlength{\textwidth}{16.0cm}        % largura do texto
\setlength{\textheight}{9.0in}        % tamanho do texto (sem head, etc)
\renewcommand{\baselinestretch}{1.15} % espaÁamento entre linhas
\addtolength{\topmargin}{-1cm}        % espaÁo entre o head e a margem
\setlength{\oddsidemargin}{-0.1cm}    % espaÁo entre o texto e a margem
       
% Ser indulgente no preenchimento das linhas
\sloppy
 

\begin{document}
\[\pagestyle {empty}

% Páginas iniciais
\documentclass[10pt]{article}
\usepackage[usenames]{color} %used for font color
\usepackage{amssymb} %maths
\usepackage{amsmath} %maths
\usepackage[utf8]{inputenc} %useful to type directly diacritic characters
\begin{document}
\[%\input logo

%\vspace*{-3cm}

%\begin{figure}[h]
%\leavevmode
%\begin{minipage}[t]{\textwidth}
%\includegraphics[1cm,1cm][3cm,3cm]{logo-ufrpe.bmp}
%\end{minipage}
%\end{figure}



\vspace*{-2cm}
{\bf
\begin{center}
{\large
\hspace*{0cm}Universidade Federal de Juiz de Fora} \\
\hspace*{0cm}Departamento de Ciência da Computação \\
\hspace*{0cm}DCC059 - Teoria dos Grafos Semestre 2016-3  \\
\end{center}}
\vspace{4.0cm}
\noindent
\begin{center}
{\Large \bf TÌtulo do Trabalho} \\[3cm]
{\Large Grupo xx: Fulano de Tal dos Anzóis Pereira}\\[6mm]
{\Large Professor: Stênio Sã Rosário F. Soares}\\[3.0cm]
\end{center}




{\raggedleft
\begin{minipage}[t]{8.3cm}
\setlength{\baselineskip}{0.25in}
Relatório do  trabalho de Teoria dos Grafos (parte 1), parte integrante da avaliação da disciplina.
\end{minipage}\\[2cm]}
\vspace{2cm}
{\center Juiz de Fora \\[3mm]
Dezembro de 2016 \\}


\newpage
\]
\end{document}           % capa ilustrativa



\pagestyle {empty}
%\abstract{ Descreva aqui o resumo do objeto de estudo do seu trabalho.}

\newpage

%\tableofcontents


% Numeração em romanos para páginas iniciais (sumários, listas, etc)
%\pagenumbering {roman}
\pagestyle {plain}



\setcounter{page}{0} \pagenumbering{arabic}
 
\setlength{\parindent}{0in}  %espaco entre paragrafo e margem 
% Espaçamento entre parágrafos
\parskip 5pt  

\section{Introdução}

%comando cria itens
\begin{itemize}
	\item Descrever a natureza do trabalho, destacando as principais funcionalidades.
\end{itemize}



\section{Metodologia utilizada}
  \begin{itemize}
    \item Descreva o que foi utilizado na solução do problema em termos de análise, algoritmos etc.
  \end{itemize}

\subsection{Estruturas de dados utilizadas}	

\subsection{Abordagens algorítmicas usadas na solução}

Nesta sub-seção, você deve explicar o funcionamento dos algoritmos. Para as funcionalidades mais complexas, é aconselhável apresentar pseudocódigos e referir-se às linhas destes na explicação.

%\section{Experimentos computacionais}
%Descreva o ambiente de testes, as instâncias usadas (citando as referências se obtidas da literatura). Considere a necessidade de apresentar testes de desempenho e e análise do comportamento dos algoritmos desenvolvidos em relaÁ„o ao crescimento das instâncias, análise esta tanto em relaÁ„o a tempo quanto a espaço. 

%… de extrema valia o uso de tabelas, gráficos etc na apresentaÁ„o dos resumos de testes.

\section{Conclusıes}
Apresente as conclusões do trabalho. Note que as conclusões do seu trabalho envolvem aspectos do problema e dos algoritmos usados.

%Incluindo referências bibliográficas
\bibliographystyle{plain} %define o estilo         
\bibliography{bibliografia} %busca o arquivo

%inserindo anexos
\appendix

\section{Anexo I}
Caso haja anexos, insira aqui.

\]
\end{document}